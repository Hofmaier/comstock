\documentclass[twoside,a4paper]{article}

\usepackage{graphicx}

\begin{document}
\section{Introduction}
\label{sec:intro}

\subsection{Recommender engines}
\label{sec:recommenderengines}

Recommender engines are services that recommend articles (items) to users based on their past actions. They attempt to infer taste and preferences and identify unknown items that are of interest.
E-commerce sites that deploy a recommender engine can have a increase in sales of 8 - 13 percent \footnote{http://www.practicalecommerce.com/articles/1942-10-Questions-on-Product-Recommendations}.

There are serveral ways to design and build a recommender eninge \cite{Dunning14}.

\begin{itemize}
\item Design a custom recommender engine. That approach requires a team of highly trained engineer and data scientist.
\item Use products that offer drag-and-drop approaches. Some of these product try to automaticaly select the right algorithms. This recommender engines aren't very effective and a most of the effort required is put into getting the data into the right format.
\item Use the service of a high-end machine-learning consultancy. These companies achieve effective results by trying a huge collection of algorithms at each problem and selecting the algoritm that gives the best result.
\end{itemize}

There are several problem and challenges in building a recommender engine:
\begin{itemize}
\item Many algorithms relies on user ratings. Ratings come from a subset of users. Only user who like to rate will rate items. 
\end{itemize}

In this article we use a simplified approach to the recommender problem described in \cite{Dunning14}. The goal of the apporach presented in this article is to provide a simple solutions that gives the user practical recommendation. There are academic approaches that produce recommendation with a smaller error but these require complex mathematical models. The goal of the apporach presented in this article is to provide a simplified, general solution for the recommender problem. In contrast to an academic approach it is easear to extend the recommender engine to a multimodel recommender.

The recommender described in this article is divided in two parts.
\begin{itemize}
\item Computation of simililarity and the update of the text search engine is done offline, ahead of time.
\item Recommendations are generated instantly by quering the text search eninge using rescents actions of the user.
\end{itemize}

\section{Co-occurence based recommender}
\label{sec:cooccurence}

In order to make recommendations, the recommenders uses items in recent user history as a query to find all items that are similar. Similar items have the items from the user history stored as indicator.s

\subsection{Log-likelihood similarity}
\label{sec:llr}

The co-occurence based recommender uses the user history in order to make recommendations. The user history doesnt contains explicit preference values. Explicit user preferences are ratings of a user for an item. The user history only contains interactions of users and items. The log-likelihood similiarity is based on the number of users (or tags) in common between two items. According to \cite{Dunning93} the log-likelihood similiarity is suitable for data that only captures the interaction and no preference between users and items. 

\section{Baseline Algorithm}
\label{sec:baselinealgorithm}

\section{Precision/Recall (Evaluation)}
\label{sec:evaluation}
Recommenders answer the question "What are the best recommendations for a user?". If we want to evaluate a recommender, we have to define, what's a "good" recommendation.
Some recommender predict the preferences of a user. One possibility to evaluate a recommender is to calculate the difference between the estimated preference and the actual preference.

Those actual prefences don't exist. Nobody knows how a user likes some new item in the future.
But we can simulate the prefencences of the future by setting aside a small part of the real data set as test data. These preferences aren't present in the training data set. Instead the recommender predicts the preferences for the missing test data set and the estimates are compared to the actual values.

Another approach to evaluate a recommender is to take a broader view of the recommender problem. It's not strictly necessary to estimate preference values in order to produce recommendations. In many cases presenting a ordered list of recommendations is sufficient. The list is ordered from best to worst recommendation.

\begin{description}
\item[Precision] Precision is the proportion of top results that relevant. Suppose the recommender recommends 5 items. If 3 items are good recommendations then the precision is $3/5$.
\item[Recall] Recall is the proportion of good recommendations that appear in the recommendation result. Suppose there are 9 good recommendations. If the recommender results contains 3 of these good recommendations then the recall is 3/9.
\end{description}

In order to calculate precision and recall the implementation determines the top \verb|n| preferences for each user. It removes those preferences from the data model. It evaluates precision and recall with the new data model. It calculates a top-N recommendation list for each user and compares it with the real top-N preferences.

The evaluation process has the following paramters:
\begin{description}
\item[Size of the result] The number of recommendations to consider. The length of the recommendation list.
\item[Relevance] Determines if an item is relevant or not.
\end{description}

\section{Multimodalrecommender}
\label{sec:multimodalrecommender}

The recommender has the follwowing parameters:
\begin{itemize}
\item The number of ratings or tags that are used for the query.
\end{itemize}

\section{Results}
\label{sec:results}


\subsection{Example data set}
\label{sec:exampledataset}

The example data set is a small set of user preferences. It has constructect properties:
\begin{itemize}
\item Items 108, 109 und 111 are similiar.
\item User 9 likes the items 111, 109 

\end{itemize}

\section{Dataset}
\label{sec:dataset}

\subsection{Movielens}
\label{sec:movielens}

The MovieLens dataset describes 5-star rating and free-text tagging activity from MovieLens, a movie recommendation service. It contains 100023 ratings and 2488 tag applications across 8570 movies. These data were created by 706 users 

All selected users had rated at least 20 movies.
All ratings are contained in the file ratings.csv. Each line of this file after the header row represents one rating of one movie by one user, and has the following format:

\begin{verbatim}
userId,movieId,rating,timestamp.
\end{verbatim}

All tags are contained in the file $tags.csv$. Each line of this file after the header row represents one tag applied to one movie by one user, and has the following format:
\begin{verbatim}
userId,movieId,tag,timestamp
\end{verbatim}

\section{Similaritity}
\label{sec:similarity}



\section{Infrastructur}
\label{sec:infrastructur}


\subsection{Apache Spark}
\label{sec:spark}
\verb|spark-rowsimilarity| is a script that comes with spark. It takes as input a text file representation of a matrix of sparse vectors. It finds similar rows. The input is in text-delimited form where there are three delimiters used. By default it reads (rowID<tab>columnID1:strength1<space>columnID2:strength2...) The job only supports LLR similarity. This job only supports LLR similarity.
The input has the following format:
\begin{verbatim}
1	1
2	2
3	3 
\end{verbatim}

\section{Case Study: artoffer.ch}
\label{sec:artoffer}

The following action are used as indicators
\begin{itemize}
\item When a user visits the page of a item. (boolean)
\item When a user likes an item.
\item Tags
\end{itemize}
\bibliographystyle{plain}
\bibliography{a}
\end{document}
