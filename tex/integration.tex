\section{Integration}
\label{sec:integration}

This section describes how we integrate the components described in section \ref{sec:design}. The section is divided according the online and offline parts of the \gls{rec}.

\begin{description}
\item[Offline learning] In this part the systems uses the stored histories of user behavior to compute the \gls{llr} ratios of the \gls{coocc} among items. Then it updates all items in the in the NoSQL database/search engine Apache Solr. We use similar items as indicators and store the \gls{llr} ratios from the previous step as payloads. This task takes serveral minutes but the duration does not impact the user experience hence it can be executed overnight. 
\item[Online logging and recommendation] The online part of the system records user activities used in the similarity analysis and creates personalized \gls{topn} if requested. If it receives a requerst for a recommendation list it constructs a query from the user history $h_u$. Then it uses the query to get a ranked list of items that are likely appealing to the user $u$. It removes all item's the user already knows and present the list as recommendation. In order to a  provide smooth user experience the top-N recommendation task should not take longer than a second.
\end{description}

To store the user's activity we used a sqlite3 database because it's easy to deploy in a development environment and we can use existing interfaces for Java and Python to update and retrieve data. 

First we insert a document for every item into Solr. The documents contain the metadata like (title, genre, tags, etc). The offline learning step will only update the indicator fields for every item.

\subsection{Offline learning}
\label{sec:offline}

\begin{figure}
\centering
\begin{tikzpicture}[node distance=20mm,
data/.style={
rectangle,
draw,
thin,
minimum height=3.5em
},
to/.style={->,>=stealth',shorten >=1pt,semithick,font=\footnotesize},]

\node (history) [data,align=left] {user actions\\database};
\node (spark) [data,right of=history,node distance=50mm, align=left] {Mahout's \\\verb|RowSimilarityJob|};
\node (solr) [data,below of=spark,node distance=3cm] {Apache Solr};
\draw[to] (history) -- node[midway,above]{user actions} (spark);
\draw[to] (spark) -- node[midway,right] {similarity indicators} (solr);
\end{tikzpicture}
\caption{The offline learning part uses user log files stored in a sqlite3 database to compute the similarities with Mahout's {\ttfamily RowSimilarityJob}. The result is loaded into Solr.}
\label{fig:offline}
\end{figure}

The offline learning part is done in three steps.

\begin{enumerate}
\item Read the recorded \glspl{useraction} from a sqlite3 database and convert it as required by \verb|RowSimilarityJob|.
\item Generate \glspl{indicator} with Mahout's \verb|RowSimilarityJob| that connects to a Apache Spark server.
\item Use the output of \verb|RowSimilarityJob| to update the indicator fields of all items stored in Solr.
\end{enumerate}

The computation of similarities and update of Solr involves a lot transforming data from one format to another. We have implemented a command line tool, \verb|updatesearchengine|, that executes all three steps. 

\subsection{Online recommendation}
\label{sec:online}

The generation of a top-N recommendation list can be divided into 4 step. The process is illustrated in figure \ref{fig:topn}.

\begin{enumerate}
\item User requests top-N recommendation list.
\item Look up the user's action history $h_u$.
\item Query the search engine.
\item Process the search result and format a JSON response.
\end{enumerate}

These steps are implemented in the web application with a few lines of code. The the heavy lifting is done by the search engine.

First the web browser requests a top-N recommendation list for user $u$. In our demo application the web server provides a REST API. A authenticated user can issue a \verb|GET| request to the path \verb|/topn| in order to get the list. 
When the web server receives a request it looks up the action history $h_u$ for user $u$ in the user action database. $h_u$ can contain multiple types of user behavior. In our demo application the web server retrieves a list of items the user liked in the past and a list of items the user tagged. In order to retrieve all actions that belong to user $u$ the user activity is stored in a relational database.
Then the web server forms a Solr query with the recent users actions, and sends Solr a HTTP request that contains the query as \verb|q| parameter. For instance the following string shows the \verb|q| parameter for a HTTP request that uses recent user likes and tag activity.
\begin{verbatim}
q=likeindicator:1688 1834 3893 AND tagindicator:10 33 41 54 55 
\end{verbatim}
Solr returns a raw ranked result list. 

To produce a top-N recommendation list the web server uses the user history $h_u$ again to removes all item already known to the user. Then it formats the data as JSON and presents the user a top-N recommendation list. In addition we could apply business logic that adjusts which items are shown. For instance, we can promote particular items or we could remove out-of-stock items.

Listing \ref{lst:topn} shows the Scala code that does the filtering and formatting.
\verb|solrResponse| is a \verb|Iterable| that yields the ranked result set from Solr. \verb|history| is a list of all items the user liked or tagged. \verb|toJson| takes an \verb|Writes[T]| as implicit parameter. Hence we transform \verb|SolrDocument| by implementing \verb|Writes[SolrDocument]|.

\begin{lstlisting}[caption={The web server removes items already known and formats the the items in JSON},label={lst:topn}]
def itemIsKnown(x: SolrDocument) =
      history.contains(x.getFieldValue("id").toString())
      
val unknownItems = solrResponse.filter {!itemIsKnown(_)}
Ok(Json.toJson(unknownItems))
\end{lstlisting}

\begin{figure}
\centering
\begin{tikzpicture}[node distance=20mm,thick,
data/.style={
rectangle,
draw,
thin,
minimum height=3.5em
},
to/.style={->,>=stealth',shorten >=1pt,semithick,font=\footnotesize},]
]

\node (web) [data, minimum height=2cm] {Web server};
\node (log) [data,above of = web,align=left,node distance=30mm] {user actions\\database};
\node (browser) [data,left of=web,node distance=7 cm,minimum height=2cm] {Webbrowser};
\node (solr) [data,below of=web,node distance=4 cm] {Apache Solr};
\draw[to] ([yshift=0.5 cm]browser.east) -- node[midway,above]{request} node[midway,below]{top-N list} ([yshift=0.5 cm]web.west);
\draw[to] ([yshift=-0.5 cm]web.west) -- node[midway,above]{top-N} node[midway,below]{recommenation list} ([yshift=-0.5 cm]browser.east);
\draw[to] ([xshift=-0.5 cm]web.north) -- node[midway,left]{get $h_u$} ([xshift=-0.5 cm]log.south);
\draw[to] ([xshift=0.5 cm]log.south) -- node[midway,right]{$h_u$}([xshift=0.5 cm]web.north);
\draw[to] ([xshift=0.5 cm]web.south) -- node[midway,right]{$h_u$ as query}([xshift=0.5 cm]solr.north);
\draw[to] ([xshift=-0.5 cm]solr.north) -- node[midway,left,align=left]{ranked list\\of similar items } ([xshift=-0.5 cm]web.south);
\end{tikzpicture}
\caption{To produce a top-N recommendation list the web browser requests a top-N recommendation list for user $u$. The webserver looks up the action history $h_u$ for user $u$ in the user action database. Then the webserver forms a Solr query with the recent users actions and sends it to Solr. Solr returns a raw ranked result list to the web server. The webserver removes all item already known to the user formats the data to JSON and presents the user a top-N recommendation list. }
\label{fig:topn}
\end{figure}
%-- node[midway,above] {post user actions}