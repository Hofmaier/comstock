\begin{abstract}
Building a recommendation engine can be a difficult and expensive task. Understanding all possible techniques and choosing the most suitable for the job at hand presupposes that you have a strong mathematical background. There are a lot different algorithms and strategies and researchers are constantly coming up with new ones. Not all companies can afford to employ an army of data scientist and mathematicens to build a recommender system. They a have to trade-off the accuracy of recommender against the development costs. 

This report describes a simple pratical approach to build a recommender system that is more approacable for small development teams. 
In order to produce a list of recommendable items for a user instantly, it uses the user' history of recent actions to query the search engine Apache Solr for similar items. 
Similarity among items is based on the \gls{llr} ratio of co-occurences of several action types (e.g. purchase, view). 
This approach exploits existing and proved technologies to save development costs. 
Further the system is scalable for big data because Solr is optimized for to search large volumes of data and co-occurence can be computed at scale with Apache Mahout's \verb|spark-itemsimilarity| job.

This report explains the core concepts of this practical approach on the basis of a small demo web applicatoin. It will explain the similarities between the weighting of indicator scores of a this model and the process of ranked retrieval.

The evaluation of the described approach shows that the accuracy of the top-N recommendation task matches the accuracy of a sophiticated itembased algorithm.
\end{abstract}
