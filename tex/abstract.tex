\begin{abstract}
Building a recommendation engine can be a difficult and expensive task. Understanding all possible techniques and choosing the most suitable for the job at hand presupposes that you have a strong mathematical background. To make matters worse there are a lot different algorithms and strategies and researchers are constantly coming up with new ones. Having said that not all companies can afford to employ an army of data scientist and mathematicens to build a recommender system. They a have to trade-off the accuracy of recommender against the development costs. 

This report describes a simple pratical approach to build a recommender system. The discussed recommender exploits the existing, proved technology of the search engine Apache Solr. It collects data about the behavior of users. Based on the past user activity it computes similarities among items and loads them into Solr. 

Co-occurence allows us to compute significant indicators of what should be recommended.
The report will discuss the similarities between the weighting of indicator scores of a this model and the process of ranked retrieval.

It uses the \gls{llr} ratio of co-occurences of several action types (e.g. purchase, view) as similarity metric. Co-occurence can be computed at scale with Apache Mahout's \verb|spark-itemsimilarity| job. 

In order to produce a list of recommendable items for a user, it uses his history of recent actions to query Solr for similar items. The approach provides a flexible and scalable way to produce recommendation in real time without to need to hire a team of datascientist. 

The evaluation of the described approach shows that the accuracy of the top-N recommendation task matches the accuracy of a sophiticated itembased algorithm.
\end{abstract}
