\newglossaryentry{topn}
{
name=top-N recommendations list,
description={The recommender engine suggest $N$ specific items that are likely appealing to him. This 'best bet' is called top-N recommendation list}
}

\newglossaryentry{indicator}
{
name={indicator},
plural={indicators},
description={Indicator are Item ID's associated with a particular item $i$. The indicators to $i$ are  ID's of similar items. Indicators are derived from user actions. They lead to other items that are recommendable for the same action. For example if you wish a user to purchase something and you collect all users purchase interactions you can create purchase indicator for every item based on these interactions}
}

\newglossaryentry{coocc}
{
name={co-occurence},
description={Co-occurence in the context of user-item interactions describes the number of shared users that interacted with a two particular items. Co-occurence can be interpreted as an indicator of similiarity between the two items}
}

\newglossaryentry{tag}
{
name={tag},
plural={tags},
description={A tag is a non-hierarchical keyword or term assigned to an item (e.g. movie). It describes an item and allows it to be found again by browsing or searching. Tags are generally chosen informally and personally by users who visit an item}
}

\newglossaryentry{scalable}
{
name={scalable},
description={Scalability is the ability of a system, network, or process to handle a growing amount of work in a capable manner or its ability to be enlarged to accommodate that growth}
}

\newglossaryentry{multimodal}
{
name={multimodal},
description={Multimodal interaction means that the user can interact in multiple modes with a system}
}

\newglossaryentry{precision}
{
name={precision},
description={Precision is the fraction of retrieved instances that are relevant}
}

\newglossaryentry{recall}
{
name={recall},
description={recall (also known as sensitivity) is the fraction of relevant instances that are retrieve}d 
}

\newglossaryentry{preference}
{
name={preference},
plural={preferences},
description={A preference is the association of a user and an item and a number expressing the strength of the user's preference for the item}
}

\newglossaryentry{llr}
{
name={log-likelihood},
description={The log-likelihood ratio is a probabilistic measure of the importance of a co-occurrence}
}

\newglossaryentry{indicatorm}
{
name={indicator matrix},
description={An indicator matrix is an $n \times n$ matrix where the values are log-likelihood ratio strengths of the co-occurences}
}

\newglossaryentry{like}
{
name={like},
plural={likes},
description={A like is a positive feedback of a user associated with an item. The user interface provides a button to like an item}
}

\newglossaryentry{useraction}
{
name={user action},
plural={user actions},
description={A user action describes the interaction of a user with the GUI. Examples are: view, purchase, like }
}
\newglossaryentry{rec}
{
name={co-occurence based recommender},
description={An itembased recommender engine that uses log-likelihood ratio of co-occurences of items in the users action history. It uses a search engine to produce recommendations. The query for the search engine are formed from the user's history }
}

\newglossaryentry{rankedretrieval}
{
name={ranked retrieval},
description={Ranked retrieval is the process of sorting document by their relevance to a query. The most relevant document are listed first}
}

\newglossaryentry{history}
{
name={action history},
description={A user's action history is the stored past behavior of a user an a website}
}

\newglossaryentry{content}
{
name={Content-based},
description={Content-based filtering methods are based on a description of the item and a profile of the user’s preference. For instance tags could be used to describe an item. For every user there is a profile to indicate the type of item this user likes. Conten-based filtering algorithm recommend items that are similar to those defined in a user's profile}
}
