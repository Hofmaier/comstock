\newglossaryentry{topn}
{
name=top-N recommendations list,
description={The recommender engine suggest $N$ specific items that are likely appealing to him. This 'best bet' is called top-N recommendation list}
}

\newglossaryentry{indicator}
{
name={indicator},
plural={indicators},
description={Indicator are Item ID associated with a particular item. Indicators are derived from user actions. They lead to other items that are recommendable for the same action. For example if you wish a user to purchase something and you collect all users purchase interactions you can create purchase indicator for every item based on these interactions}
}

\newglossaryentry{coocc}
{
name={co-occurence},
description={Co-occurence in the context of user-item interactions describes the number of shared users that interacted with a two particular items. Co-occurence can be interpreted as an indicator of similiarity between the two items}
}

\newglossaryentry{tag}
{
name={tag},
plural={tags},
description={A tag is a non-hierarchical keyword or term assigned to an item (e.g. movie). It describes an item and allows it to be found again by browsing or searching. Tags are generally chosen informally and personally by users who visit an item}
}

\newglossaryentry{scalable}
{
name={scalable},
description={Scalability is the ability of a system, network, or process to handle a growing amount of work in a capable manner or its ability to be enlarged to accommodate that growth}
}

\newglossaryentry{multimodal}
{
name={multimodal},
description={Multimodal interaction means that the user can interact in multiple modes with a system}
}

\newglossaryentry{precision}
{
name={precision},
description={Precision is the fraction of retrieved instances that are relevant}
}

\newglossaryentry{recall}
{
name={recall},
description={recall (also known as sensitivity) is the fraction of relevant instances that are retrieved}
}

\newglossaryentry{preference}
{
name={preference},
description={A preference is the association of an user and an item and a number expressing the strength of the user's preference for the item}
}
