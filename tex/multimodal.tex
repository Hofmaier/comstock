\subsection{Using more than one type of behavior}
\label{sec:multimodal}

Most collaborative filtering algorithms use only explicit or implicit ratings to compute similarity. Hence they only use one type of user behavior.
But we can improve the performance of the recommender engine by using multiple types of user actions. As described in section \ref{sec:inputdata} we record tag-associations. In addition to likes we can use tagging activity to compute similarity. In table \ref{tbl:llr1} we count the co-occurrence of items in user's like-action history. Instead of the like history we could use tags that are associated with items. We count the \gls{coocc} of each items with respect to item-tag associations.

Suppose we compute one \gls{indicatorm} based on likes $M_l$ and one based on tag associations $M_t$. And $h_l$ is a user's history of ``likes'' and $h_t$ is the user's tag history. We use both indicators to compute the recommendation vector $r$ by first calculation the matrix vector products of a indicator matrix with its corresponding history vector and then summing up the resulting vector (see equation \ref{eq:multi}).

\begin{equation}
  \label{eq:multi}
  r = h_l M_l + h_t M_t
\end{equation}

In our demo web application we use ``likes'' and tags but virtually all user actions can be used to improve the recommendations.
