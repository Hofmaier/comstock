\subsection{Using more than one type of behavior}
\label{sec:multimodal}

Most collaborative filtering algorithms use only explicit or implicit ratings to compute similarity.
But we can improve the performance of the recommender engine by using multiple types of user actions. In addition to likes we could use tag-associations to compute the similariy. In table \ref{tbl:llr} we count co-occurence of items in a user's like-action history. Instead of the action history we could use tags that are associated width items. We count the co-occurence of each items associated with a tag.

Suppose we compute a \gls{indicatorm} based on likes $M_l$ and one based on tag associations $M_t$. An $h_l$ is a user's history of ``likes'' and $h_t$ is the user's tag history. Then we can compute the recommendation vector (see equation \ref{eq:recommendation}) $r$ with

\begin{equation}
  \label{eq:multi}
  r = h_l M_l + h_t M_t
\end{equation}

In our demo web application we use ``likes'' and tags but virtually all user actions can be used to improve the recommendation.
