\subsection{Record user behavior}
\label{sec:inputdata}

\begin{figure}
  \centering
     \includegraphics[width=0.9\textwidth]{collectinginput}
  \caption{The user can like and tag movies with the web front end. The user actions are recorded by the web server.}
  \label{fig:gui}
\end{figure}

The \gls{rec} suggests items that are similar to the ones the user already liked in the past. In order to build a model for similarity we need to train the recommender with some data about the items.
This section will descibe the type of input data we use  in our demo application. The report will refer to this type of input data. 

Many collaborative filtering recommender engines use excplicit user ratings to train their model. Explicit user ratings of a user for an item are expressed by numbers (e.g. a rating is a number between 1 and 5). The use of explicit feedback has some drawbacks.
\begin{itemize}
\item Only a small subset of users will rate items. This leads to a model that is skewed against user who like to rate.
\item The majority of ratings are associated with a small fraction of the most popular items \cite{Anderson}. As a result it is less likely that unknown items show up in the \gls{topn}. This behavior is undesirable because the goal of the recommender is to present items the a user would not find on his own.
\end{itemize}

Corresponding to \cite{Dunning14} the best choice of input data is the collection past user actions on a website. The stored behavior of one user is called the user's \gls{history}. It shows what users actually do. Hence the input data should consist of recorded \glspl{useraction} (e.g. purchase, view, like, tag).

In our demo web application we record two different user actions:
\begin{description}
\item[like]  Users can express their positive feedback for a movie by clicking on a ``like'' button (the \gls{like} action is an explicit rating. We use it instead of a purchase or view action in order to keep the GUI simple).
\item[tag] User can \gls{tag} items. Every item can be associated with a list of \glspl{tag}.
\end{description}
The recorded like and tag action are later used to compute similarity between items.
Figure \ref{fig:gui} shows the simplistic user interface of our demo web app.

The web browser sends every user action to the web server. The web server provides a REST Web API that receives the \glspl{useraction} as HTTP \verb|Post| request and saves them to a sqlite3 \footnote{https://www.sqlite.org/} database.

In order to analyse the data later we want to retrieve the action history $h_u$ for a particular user $u$ for a defined action and a list of tags for every item. Hence we have to structure the data accordingly. Figure \ref{fig:er} shows the entity relationship diagram for the user actions \gls{like} and \gls{tag}

\tikzset{multi  attribute/.style={attribute ,double  distance=1.5pt}}
\tikzset{derived  attribute/.style={attribute ,dashed}}
\tikzset{total/.style={double  distance=1.5pt}}
\tikzset{every  entity/.style={draw=blue , fill=blue!20}}
\tikzset{every  attribute/.style={draw=yellow, fill=yellow!20, node distance=1.0cm}}
\tikzset{every  relationship/.style={draw=red, fill=red!20}}

\begin{figure}
\centering
\begin{tikzpicture}[node distance=2.0cm]
  \node[entity](user){user};
  \node[relationship](like)[above right of=user]{ like } edge (user);
  \node[attribute](date2)[above of=like]{ date } edge (like);
  \node[entity](item)[below right of=like]{item} edge (like);
  \node[relationship](tag)[below right of=user]{ tag } edge (user) edge (item);
  \node[attribute](text)[below right of=tag]{ text } edge (tag);
  \node[attribute](date1)[below left of=tag]{ date } edge (tag);
  \end{tikzpicture}
\caption{The data model allows us to retrieve the action history for every user. Like and tag form associations of user with item.}
\label{fig:er}
\end{figure}

It makes sense to start recording behavioral data month's before depoying the recommender engine because the recommender engine has to analyze the data and build a model of similarty among items in order to create personalized recommendations.

