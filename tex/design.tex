\section{Co-occurence based multimodal recommender}
\label{sec:design}
\begin{figure}
  \centering
  \begin{tikzpicture}
    \node (rec) at(180:2cm) {Recommender};
    \node (hist) at(60:2cm){History};
    \node (se) at(300:2cm){Search engine};
    \node (user) [left of=rec,node distance=3cm]{User};
    \draw[->, >=latex] (170:2cm) 
    arc (170:80:2cm);
  \end{tikzpicture}
  \caption{Simplified dataflow diagram}
  \label{fig:topndataflow}
\end{figure}

The recommender we discuss will use past user behavior and metadata of items to compute the similarity between all items. Hence it is a hybrid recommender.
We use \gls{coocc} to compute the similarity of two items. Because search engine are a way of finding similar items we deploy the recommender queries a search engine in order to find similar items to the one the user already exressed some interest.

The recommender discussed in this article has to parts.
\begin{description}
\item[Analysing User Input (offline)] In this part the user action are analised in order to compute the similarities of items. The similiarities are stored as indicators.
\item[Generate personalized recommendations (online)] A systems formats a list of recommended items.
\end{description}


The process of procucing a \gls{topn} can be divided into several steps. Figure \ref{fig:topndataflow} illustrates this process.

\begin{enumerate}
\item User requests a \gls{topn}.
\item The recommenders looks up items that appear in the recent user history.
\item Based on the user' action history the recommender forms a for the search engine query.
\item The search engine return ranked list of items according to the query.
\item The recommender removes item already known to the user and present him a \gls{topn}.
\end{enumerate}

All movies are stored as documents in a NoSQL database (we use Apache Solr). The documents contain metainformation (e.g. title, tags, genre) about the items in fields. The fields are indexed by a search engine and made searchable.

In addidion the document has indicator fields. Indicator fields contain id's of that are found to be worth recommending in the co-occurence analysis.

The section first describes the used input data. Than we explain the process of computing \gls{topn} and how similarity between items is measured and how the computation is implemented

\subsection{Record user behavior}
\label{sec:inputdata}

\begin{figure}
  \centering
     \includegraphics[width=0.9\textwidth]{collectinginput}
  \caption{The user can like and tag movies with the web front end. The user actions are recorded by the web server.}
  \label{fig:gui}
\end{figure}

The \gls{rec} suggests items that are similar to the ones the user already liked in the past. In order to build a model for similarity we need to train the recommender with some data about the items.
This section will descibe the type of input data we use  in our demo application. The report will refer to this type of input data. 

Many collaborative filtering recommender engines use excplicit user ratings to train their model. Explicit user ratings of a user for an item are expressed by numbers (e.g. a rating is a number between 1 and 5). The use of explicit feedback has some drawbacks.
\begin{itemize}
\item Only a small subset of users will rate items. This leads to a model that is skewed against user who like to rate.
\item The majority of ratings are associated with a small fraction of the most popular items \cite{Anderson}. As a result it is less likely that unknown items show up in the \gls{topn}. This behavior is undesirable because the goal of the recommender is to present items the a user would not find on his own.
\end{itemize}

Corresponding to \cite{Dunning14} the best choice of input data is the collection past user actions on a website. The stored behavior of one user is called the user's \gls{history}. It shows what users actually do. Hence the input data should consist of recorded \glspl{useraction} (e.g. purchase, view, like, tag).

In our demo web application we record two different user actions:
\begin{description}
\item[like]  Users can express their positive feedback for a movie by clicking on a ``like'' button (the \gls{like} action is an explicit rating. We use it instead of a purchase or view action in order to keep the GUI simple).
\item[tag] User can \gls{tag} items. Every item can be associated with a list of \glspl{tag}.
\end{description}
The recorded like and tag action are later used to compute similarity between items.
Figure \ref{fig:gui} shows the simplistic user interface of our demo web app.

The web browser sends every user action to the web server. The web server provides a REST Web API that receives the \glspl{useraction} as HTTP \verb|Post| request and saves them to a sqlite3 \footnote{https://www.sqlite.org/} database.

In order to analyse the data later we want to retrieve the action history $h_u$ for a particular user $u$ for a defined action and a list of tags for every item. Hence we have to structure the data accordingly. Figure \ref{fig:er} shows the entity relationship diagram for the user actions \gls{like} and \gls{tag}

\tikzset{multi  attribute/.style={attribute ,double  distance=1.5pt}}
\tikzset{derived  attribute/.style={attribute ,dashed}}
\tikzset{total/.style={double  distance=1.5pt}}
\tikzset{every  entity/.style={draw=blue , fill=blue!20}}
\tikzset{every  attribute/.style={draw=yellow, fill=yellow!20, node distance=1.0cm}}
\tikzset{every  relationship/.style={draw=red, fill=red!20}}

\begin{figure}
\centering
\begin{tikzpicture}[node distance=2.0cm]
  \node[entity](user){user};
  \node[relationship](like)[above right of=user]{ like } edge (user);
  \node[attribute](date2)[above of=like]{ date } edge (like);
  \node[entity](item)[below right of=like]{item} edge (like);
  \node[relationship](tag)[below right of=user]{ tag } edge (user) edge (item);
  \node[attribute](text)[below right of=tag]{ text } edge (tag);
  \node[attribute](date1)[below left of=tag]{ date } edge (tag);
  \end{tikzpicture}
\caption{The data model allows us to retrieve the action history for every user. Like and tag form associations of user with item.}
\label{fig:er}
\end{figure}

It makes sense to start recording behavioral data month's before depoying the recommender engine because the recommender engine has to analyze the data and build a model of similarty among items in order to create personalized recommendations.


\subsection{How to compute the top-N recommendations list?}
\label{sec:problem}
This section gives a mathematical description of the top-N recommendation task and hence it describes the computations required to produce recommendations. The description refers to the \gls{rec}.

Suppose we have a metric to express similarity between two items as a numerical value and the magnitude of the value determines the strength of similarity. If we compute the similarity for every item pair in a set of $n$ items, we can represent the result in a matrix $M$. $M$ is a $n \times n$ matrix. Each row and each column contains the similarities between one particular item and all other items. $M$ is symetric across the diagonal because the similarity between $a$ and $b$ must be the same as between $b$ and $a$ (commutativity). The diagonal of $M$ contains the value for maximum similarity because this value represents the comparison of an item to itself. $M$ is called the \gls{indicatorm}. Equation \ref{eq:similaritymatrix} shows an example of an \gls{indicatorm} with 4 items.

\begin{equation}
  \label{eq:similaritymatrix}
M =\bordermatrix{~ & 1 & 2 & 3 & 4 \cr
 1 & 1  & 0.40 & 0.9 & 0.1 \cr
2 & 0.40 &1  & 0.9 & 0.1 \cr
 3& 0.9 & 0.9 &1  & 0.63 \cr
 4 & 0.1 & 0.1 & 0.63 &1  \cr}
\end{equation}
Further we represent a user action history for each user as vector $h_l$ of length $n$. $h_l$ contains an element for every item. The user's interactions with an item are represented as binary values in $h_l$. If the there is an interaction with an item in the history the value or the corresponding element is 1. Otherwise the value is 0. For example equation \ref{eq:history} shows a user's action history for the action ``like''. He has liked item 1 and 2.

\begin{equation}
\label{eq:history}
h_l =
\begin{pmatrix}
 1 \\
 1 \\
 0 \\
 0 \\
\end{pmatrix}
\end{equation}

To create a \gls{topn} for user $u$ we compute the matrix vector product of $M$ and $h_l$. The result $r$ is a vector of length $n$, that contains a value for every item. $r$ maps every item to a value that indicates how likely an item is of interest to user $u$. According to equation \ref{eq:recommendation} item 3 correspond to the best recommendation.

\begin{align}
  \label{eq:recommendation}
r_u &= M h_u 
&=
\begin{pmatrix}
  1  & 0.40 & 0.9 & 0.1 \\
 0.40 &1  & 0.9 & 0.1 \\
  0.9 & 0.9 &1  & 0.63 \\
  0.1 & 0.1 & 0.63 &1 \\  
\end{pmatrix} 
\begin{pmatrix}
 1 \\
 1 \\
 0 \\
 0 \\
\end{pmatrix}
&= 
\begin{pmatrix}
 1.4 \\
 1.4 \\
 1.8 \\
 0.2 \\
\end{pmatrix}
\end{align}

In order to create the complete \gls{topn} based on the vector $r$ we create a list of all item sorted by the values in $r$. Items with a high value appear first in the list. Then we remove all items from the list the user hasn't seen (the ones with zeros in $h_u$). In other words we return a ranked list of items. This list forms the \gls{topn}. In the example of equation \ref{eq:recommendation} the recommender would return item 3 followed by item 4. Item 1 and to are removed because the apear in the user's history.

\subsection{How to measure similarity among items?}
\label{sec:llr}

In the last section (\ref{sec:problem}) we use a matrix $M$ that contains similarity strengths among items. This section describes how $M$ is computed and why the \gls{llr} ratio of \glspl{coocc} is suitable for a recommender engine.

In order to compute the similarity between two items we count the \gls{coocc} among two items with respect to a particular user action and then compute the the \gls{llr} ratio of that \gls{coocc}.

\subsubsection{Co-occurrence}
\label{sec:cooccurence}

Co-occurrence in the context of a recommender system is the number of times a pair of items appear together in some user's action history or another item-interaction (e.g. tag-item). For instance, if there are 5 users who all liked items $A$ and $B$ then $A$ and $B$ co-occur 5 times. \Gls{coocc} indicates similarity. The more two items turn up together, the more related they probably are. We can count the \gls{coocc} of items with respect to any action or entity. For instance, we can count how many times two items are associated with the same tag or purchased by the same users.

\begin{figure}
\centering
\begin{tikzpicture}[node distance=40mm,
data/.style={
rectangle,
draw,
thin,
minimum height=3.5em
},
to/.style={->,>=stealth',shorten >=1pt,semithick,font=\footnotesize},
]
\node (hist) [data, align=left] {User actions\\history};
\node (co) [data,right of=hist,align=left] {Co-occurence};
\node (in) [data,right of=co,align=left] {LLR\\indicator\\matrix};
\draw[to] (hist) -- (co);
\draw[to] (co) -- (in);
\end{tikzpicture}
\caption{To compute the indicator matrix we first count the co-occurrences of items and then we compute the log-likelihood strengths of the \glspl{coocc}.}
\label{fig:llrworkflow}
\end{figure}

\subsubsection{Log-likelihood ratio}
\label{sec:llrs}

\Gls{llr} is a probabilistic measure of the importance of a \gls{coocc}. The \gls{llr} similarity  is the probability that two users share the same items because the items are similar and not due to chance. It finds important \glspl{coocc} and filters out the coincidental. Hence it avoids that the result is skewed against popular items \cite{Dunning93}. Compared to the Jaccard coefficient \cite{Hartung} the log-likelihood-based similarity computes higher similarities for anomalous co-occurrences than for items that occur in every user history. For a detailed explanation of the math involved see \cite{Dunning93}. 

According to \cite{Dunning14} using the \gls{llr} ratio of the \gls{coocc} has several advantages.
\begin{itemize}
\item It yields good results for data that only captures the interaction and no explicit numerical \glspl{preference} value \cite{Dunning93}.
\item The similarity is not skewed against popular items.
\item We can use distributed MapReduce based algorithms to compute the \glspl{coocc}. Hence the computation of the \gls{llr} similarity is \gls{scalable}.
\end{itemize}

\subsubsection{Example}
\label{sec:llrexample}

We describe the log-likelihood based similarity with a small example data set. Suppose we analyze the user action history for the action ``like'' given in table \ref{tbl:llr1}. 
Table \ref{tbl:llr1} shows the likes of four users for five items. The items are represented with integers 1-4 and the users with integers 101 - 104  (see appendix listing \ref{lst:sampledata} for raw web log).
In the example data set of table \ref{tbl:llr1} the items 1 and 2 are similar because three users liked both of them.

We compute the \gls{indicatorm} in two steps as shown in figure \ref{fig:llrworkflow}.
\begin{enumerate}
\item Count \glspl{coocc}
\item Compute \gls{llr} of \glspl{coocc}
\end{enumerate}

\begin{table}
\begin{center}
\begin{tabular}{rllll}
 & 101 & 102 & 103 & 104\\
1 & x & x & x &  \\
2 & x &   & x & x\\
3 & x & x & x &  \\
4 &   & x & x & x\\
5 & x & x & x & x\\
\end{tabular}
\end{center}
\caption{Example data set. The columns represent the user interaction with an item. Items are named 1 - 4 and users 101 - 104}
\label{tbl:llr1}
\end{table}

In order to get the similarities between all items we count the \gls{coocc} of ``\glspl{like}'' for all item pairs. This leads to the $5 \times 5$ \gls{indicatorm} $C$ shown in equation \ref{eq:coocm}. The rows and the columns are items. $C$ is a similarity comparison of every row of table \ref{tbl:llr1} to every other row.

\begin{equation}
  \label{eq:coocm}
C =\bordermatrix{~ & 1 & 2 & 3 & 4 & 5 \cr
1 & 4 & 2 & 3 & 2 & 3 \cr
2 & 2 & 3 & 2 & 1 & 3 \cr
3 & 3 & 2 & 3 & 2 & 3 \cr
4 & 2 & 1 & 2 & 3 & 3 \cr
5 & 3 & 3 & 3 & 3 & 4 \cr}
\end{equation}

In the next step we compute the \gls{llr} ratio strength of the \glspl{coocc} for every item pair. This will again produce a $5 \times 5$ \gls{indicatorm}. Equation \ref{eq:coocm1} shows the \gls{indicatorm} for the sample data set from table \ref{tbl:llr1}.

\begin{equation}
  \label{eq:coocm1}
L =\bordermatrix{~ & 1 & 2 & 3 & 4 & 5 \cr
1 &   & 0.40 & 0.81 & 0.63 & 0 \cr
2 & 0.40 &  & 0.40 & 0.63 & 0 \cr
3 & 0.81 & 0.40 &  & 0.63 & 0 \cr
4 & 0.63 & 0.63 & 0.63 &  & 0 \cr
5 & 0 & 0 & 0 & 0 & \cr
}
\end{equation}

Although item 5 shares all users with item 1 and 3, the log-likelihood ratio is 0 because every user purchased item 5. The goal of collaborative filtering is to show the user items he would not find by himself. Item 5 is popular and a user will probably discover it by looking up a list of items sorted by popularity (this is a form of non-personalized recommendation). Hence Item 5 is not a valuable personal recommendation because we could extract it without a recommender. For this reason the \gls{llr} is suitable similarity metric for a recommender engine.

\subsubsection{Log-likelihood similarity implementation}
\label{sec:llrimpl}

Apache Mahout provides an implementation of log-likelihood similarity with the class \verb|LogLikelihoodSimilarity|. Unfortunately the LogLikelihoodSimilarity is a non-distributed implementation. It would take too long to calculate the indicator matrix for a data set with over 10 million items and we would have difficulties to load all data into the memory. 

The computation of the \gls{coocc} of every item pair can be distributed and run in parallel by applying the MapReduce programming model as follows:
\begin{description}
\item[Map] Determine all \glspl{coocc} for one user's history and yield a pair of items for each \gls{coocc}
\item[Reduce] For each item collect all corresponding item pairs of the map phase and count all \glspl{coocc} and yield a vector with all items and the corresponding \gls{coocc}.
\end{description}

This task can run in parallel on different nodes on a cluster computer framework, such as Apache Spark. Hence the computation of the \gls{indicatorm} is \gls{scalable}.
In order to compute the \gls{llr} similarity distributed on a Spark cluster, Apache Mahout provides the \verb|spark-item-| \verb|similarity| job. 
\verb|spark-itemsimilarity| is a command line job and we can start it from the Mahout shell.
\begin{verbatim}
./mahout spark-itemsimilarity --input $infile --output $outfile
\end{verbatim}
The job connects to the Spark cluster instance defined by the environment variable \verb|MASTER| and computes the \gls{indicatorm} in parallel. With the \verb|spark-itemsimilarity| job the indicator matrix can be computed in $O(n)$ \cite{Schelter}. 
The input text file contains a row for every user-item interaction. They have to be in the following format:
\begin{verbatim}
userID, action, itemID
\end{verbatim}
The output will be a text file that represents the indicator matrix as sparse vectors for every item. For every item we get the similarities to all other items.
\begin{verbatim}
itemID1<tab>itemID2:similarityvalue<space>itemID3:simvalue...
\end{verbatim}

In our demo application we have written a Python script to fetch the data from the sqlite3 database and transform to the Spark input format.



\subsection{Using more than one type of behavior}
\label{sec:multimodal}

Most collaborative filtering algorithm use only explicit or implicit ratings to compute similarity.
But we can improve the performance of the recommender engine by using multiple types of user actions. In addition to likes we could use tag-associations to compute the similariy. In table \ref{tbl:llr} we count co-occurence of items in a user's like action history. Instead of the action history we could use tags that are associated items. We count the co-occurence of items associated with a tag.

Suppose we compute a \gls{indicatorm} based on likes $M_l$ and one based on tag associations $M_t$. An $h_l$ is a user's history of ``likes'' and $h_t$ is the user's tag history. Then we can compute the recommendations $r$ with

\begin{equation}
  \label{eq:multi}
  r = h_l M_l + h_t M_t
\end{equation}

In our demo web application we use ``likes'' and tags but virtually all user actions can be used to improve the recommendation.

\subsection{Why can we use a search engine to produce \gls{topn}?}
\label{sec:relation}

The \gls{rec} uses a search engine to produce a \gls{topn}. We can deploy a search engine in order to provide recommendations because there are similarities between the computation of top-N recommendation task and the retrieval or a ranked search result set.
 This section explains why the deployment of a search engine is suitable for the top-N recommendation task.

A search engine enables user to search a collection of documents for specified keywords in a query.It returns an sorted set of documents that match the query. The result set is sorted by relevancy. The top documents are the most relevant to the query. This process is called \gls{rankedretrieval}. It does this by calculating a similarity scorbetween each document and the query and then sorts the result by this score. The score indicates the strength of the match against the query. This is one of the main use cases where search engines shine compared to relational databases. There a row either matches a query or it does not. 

One way to calculate the similarity between a query and a document is to use the vector space model.
In the vector space model each document $d$ and the query $q$ are represented as a vectors $\vec{v}(d)$ and $\vec{v}(q)$. The vector contains an element for each term. It maps every term $t$ to a tf-idf weight. tf-idf reflects how important a term is to document in the collection (see \cite{Manning} for a detailed description). 
The similarity score between two items is equal to the dot product.
\begin{equation}
  \label{eq:score}
  \text{score}(d,q) = \vec{v}(d) \cdot \vec{v}(q)
\end{equation}

We represent the collection of document as matrix $M$. Each row of $M$ is a document vector. If we want to compute all relevancy scores the search engine would compute the matrix vector product of $M$ and $q$.
\begin{equation}
  \label{eq:ser}
  r = M q
\end{equation}

This is similar to the computation for the \gls{topn} described in section \ref{sec:problem}. If we replace the query $q$ with the user's action history $h_u$ and the documents vectors  $\vec{v}(d)$with item \gls{llr} indicators the search engine will return a top-N recommendation list. All we have to do is remove the items allready known to the user.

\todo{tabell einfuegen mit vergleich search engine fuer documents similarity engine fuer item}o

\begin{table}
\begin{center}
\begin{tabular}{ll}
 search engine & \gls{rec}\\
 document & item\\
 query & user's action history \\
term & itemid \\
\end{tabular}
\end{center}
\caption{Comparison of }
\label{tbl:llr}
\end{table}

In addition 

A common approach to users for coming up with good queries is to think of words that would likely appear in a relevant document, and to use those words as query.

Let's say we represent the recommendations for a user as a vector $r$. The elements of $r$ are floating point number which represent preferences for all items for a user. Most collaborative filtering type recommenders compute $r$ by multiplying the given preferences of a user $h_u$ with the indicator matrix $M$ for all items. In our example $M$ contains the similarity values of the log-likelihood cooccurence.

\begin{equation}
  \label{eq:cf}
  r = h_u M
\end{equation}

Equations \ref{eq:cf} actually means to compare the user history $h_u$ to the rows of the indicator matrix $M$. This result in a vector $r$ containing a score that indicates the strength of the match of the item to the history $h_p$. The recommender ranks the items by the score and presents them to the user. These items form the recommendations.

This is exactly what the ranked retrieval feature of a search eninge does.
The user history $h_U$ is the query. The items are the documents. And the text of the fields contain similar items.

In addition we can use TF-IDF \cite{Manning} weighting to mitigate popular items.

This is why we deploy Solr to build a recommender.

We store all items as documents in Solr. The documents contain the metadata like (title, genre, tags, etc). In addidtion we populate a filed for every indicator with the similar item ID's discovered with the coocuccence similartiy from section \ref{sec:llr}.

\begin{lstlisting}[caption={Item metadata and similar items are stored in Solr.},label={lst:solrdoc}]
{
    "id": "1",
    "title": ["Toy Story (1995)"],
    "tags":"Pixar animation fantasy",
    "likeindicator": "1688 1834 3893 4366 6281 33162 50872 53000 ",
    "_version_": 1505056335358591000
}
\end{lstlisting}

In order to build a recommender using a search engine we store the output of the co-occurence analysis in Solr. The search engine actually delivers the recommendations to our users.

\subsection{Retrieve recommendation}

In order to produce recommendations we compose a Solr query from the user history. The user history is stored in the web log. The web server sends this query to Solr. Solr responds with a ranked result set. The web server then formats the response from Solr and sends a list of recommended items to the user.

\begin{figure}
\centering
\begin{tikzpicture}[node distance=20mm,
data/.style={
rectangle,
draw,
thin,
minimum height=3.5em
},
to/.style={->,>=stealth',shorten >=1pt,semithick,font=\footnotesize},
]
\node (web) [data] {Web server};
\node (log) [data,below of = web, align=left] {User actions\\log file};
\node (browser) [data,left of=web,node distance=50mm] {Webbrowser};
\node (solr) [data,right of=web,node distance=50mm] {Search engine};
\draw[to] (web) -- (log);
\draw[to] (browser) -- node[midway,above] {user actions} (web);
\end{tikzpicture}
\caption{The web server sends this query to Solr. Solr responds with a ranked result set.}
\end{figure}


\subsection{Integration}
\label{sec:integration}

\verb|updatesearchengine| will index all movies in Solr.

Solr is used in the offline and the online part of the recommendation engine.

The items and their corresponding similarity indicators from the Apache Spark job are stored with Apache Solr. 

\subsection{Parameters}
\label{sec:parameters}

This section descripes the paratemter of the recommender discussed in this report.
\begin{description}
\item[similarity threshold] We have to define a threshold to separate similar items occording to the LLR similarity from the rest (e.g. 0.5).
\item[user history to consider] We retrieve recommendations with a part of the user history. We have to define the number of log entries to consider.
\end{description}

\subsection{Two-parts design}

The recommender described in this article is divided in two parts.
\begin{itemize}
\item Computation of simililarity and the update of the text search engine is done offline, ahead of time.
\item Recommendations are generated instantly by quering the text search engine using rescents actions of the user.
\end{itemize}
