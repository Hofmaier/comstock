% Created 2015-03-17 Die 22:32
\documentclass[11pt]{article}
\usepackage[utf8]{inputenc}
\usepackage[T1]{fontenc}
\usepackage{fixltx2e}
\usepackage{graphicx}
\usepackage{longtable}
\usepackage{float}
\usepackage{wrapfig}
\usepackage{soul}
\usepackage{textcomp}
\usepackage{marvosym}
\usepackage{wasysym}
\usepackage{latexsym}
\usepackage{amssymb}
\usepackage{hyperref}
\tolerance=1000
\providecommand{\alert}[1]{\textbf{#1}}

\title{Project Proposal}
\author{Lukas Hofmaier}
\date{\today}
\hypersetup{
  pdfkeywords={},
  pdfsubject={},
  pdfcreator={Emacs Org-mode version 7.9.3f}}

\begin{document}

\maketitle

\section{Problem description}
\label{sec-1}

The website artoffer.ch provides ca. 12000 artists a platform to present their works of art to broad audience on the web. 
Visitors can discover the art of artists and get in contact with them. 
It is also possible to buy pieces of art. 
For visitors it’s hard to find pieces of art, which matches their taste because there are too many of them (100000 items).

Collaborative filtering is a method of making automatic predictions (filtering) about the user preferences or taste.
To compute the prediction collaborative filtering is collecting ratings or behavior information from many users from the past. 
Model-based collaborative filtering describes user and items as a set of latent factor vectors. 
The inner product can be used to predict the interest of a user for an item.
The method can be applied to present the visitors of artoffer.ch pieces of art that could interest them.

Building the latent factor model is expensive. 
As the numbers of users and items grow, model-based collaborative filtering will suffer serious scalability problems. 
Cloud computing services, like Microsoft Azure, satisfy the demand of a higher scalability. 
\section{Projectgoals}
\label{sec-2}

\begin{itemize}
\item Study of cloud computing and the fundamentals of the technologies Microsoft Azure, Azure Machine Learning and Apache Spark
\item Comparison of Microsoft Azure Machine Learning and Apache Spark for developing and executing a CF algorithm on a large dataset. 
\item Analysis and usage of implicit feedback of the visitors
\item Build a recommender for artoffer.ch
\item A technical report containing the theoretical and practical results.
\end{itemize}


\begin{center}
\begin{tabular}{ll}
 Projectstart    &  March 2015      \\
 Projectend      &  September 2015  \\
 Semester        &  FS 2015         \\
 Projectpartner  &  ?               \\
 Advisor         &  ?               \\
 Titel           &  ?               \\
\end{tabular}
\end{center}

\end{document}
